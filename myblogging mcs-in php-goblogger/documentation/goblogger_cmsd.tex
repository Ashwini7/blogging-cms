\documentclass{article}
\title{Go Blogger}
\author{ {\sc }\\Created by Ashwini Kumar\\PRN:[11030142035]\\ashwini7security@gmail.com
\\Symbiosis Instituue Of Computer Studies \& Research\\}
\date{\today}
\usepackage{graphics,graphicx}
\usepackage{hyperref}

\begin{document}
\maketitle
\section{\bf Introduction:}
Go Blogger is a free software open source  package written in PHP, originally for use on Blogging Site.  It runs on Most operating systems, is written in PHP, primarily uses the MySQL database server.\\

 It is not a tutorial; it just points you to various places where you can go learn whatever is necessary. \\


\section{\bf What is Blog ?}
\begin{itemize}
 \item A blog is a personal diary. A daily pulpit. A collaborative space. A political soapbox. A breaking-news outlet. A collection of links. Your own private thoughts. 
\subsection{Blog in Detail !}
A blog is basically a type of website, like a forum or a social bookmarking site. As such it is defined by the technical aspects and features around it, and not by the content published inside it.

The features that make blogs different from other websites are:

\item Content is published in a chronological fashion
\item Content is updated regularly
\item Readers have the possibility to leave comments
\end{itemize}

\section{Installation Of Go Blogger !}

\subsection {\bf Requirements:} 
\begin{itemize}
\item Download Go Blogger \\
\item Web server such as Apache\\
\item PHP. 
\item Database Server \\
(a) MySQL 5.1 
\end{itemize}

\section{\bf Editing Pages !}
\begin{itemize}
\item It's very easy to write the blog . It only takes a few clicks.\\
1. Click the "blog" page tab at the top of the main page.\\
2. Make changes to the text.\\
3. Click the "Publish" button.\\

\item Simple as that! 
\end{itemize}

\subsection{\bf How To Use Go Blogger !}
\begin{itemize}
\item First copy paste go blogger in your document root.
 \item Secondly run createdb.php -- Which will create your database and tables in your database.
\item  Run index.html file in your browser.And use go blogger.
\end{itemize}
\subsection{\bf installaions images}

\begin{figure}[h]
 \scalebox{0.2}{\includegraphics{}}
  \caption{Portfolio Page Image}
\end{figure}

\end{document}